% !TEX encoding = UTF-8 Unicode

%
% Exemple de rapport
% par Pierre Tremblay, Universite Laval
% modifié par Christian Gagne, Universite Laval
% modifié par Francis Valois, Université Laval
% 31/01/2011 - version 1.4
%

%
% Modele d'organisation d'un projet LaTeX 
% rapport/      dossier racine et fichier principal
% rapport/fig   fichiers des figures
% rapport/tex   autres fichiers .tex
%

% ** Preambule **
%
% Ajouter les options au besoin :
%    - "ULlof" pour inclure la liste des figures, requis si "\begin{figure}" utilise
%    - "ULlot" pour inclure la liste des tableaux, requis si "\begin{table}" utilise
%
\documentclass[12pt,ULlof,ULlot]{ULrapport}

% Chargement des packages supplementaires (si absent de la classe)
\usepackage[utf8]{inputenc}
\usepackage[T1]{fontenc}
\usepackage[autolanguage]{numprint}
\usepackage{icomma}
\usepackage{subfigure}
\usepackage{graphicx}
\usepackage[absolute]{textpos}
\usepackage[final]{pdfpages}
\def\dbar{{\mathchar'26\mkern-12mu d}} 
%\usepackage[options]{nom_du_package}

% Definition d'une commande pour presenter des cellules multilignes dans un tableau
\newcommand{\cellulemultiligne}[1]{\begin{tabular}{@{}c@{}}#1\end{tabular}}


% Definition de colonnes en mode paragraphe avec alignement ajustable
% Cette definition requiert le chargement du package "array"
%    - alignement horizontal, parametre #1 : - \raggedright (aligne a gauche)
%                                            - \centering (centre)
%                                            - \raggedleft (aligne a droite)
%    - alignement vertical, parametre #2 : - p (aligne en haut)
%                                          - m (centre)
%                                          - b (aligne en bas)
%    - largeur, parametre #3 : longueur
\newcolumntype{Z}[3]{>{#1\hspace{0pt}\arraybackslash}#2{#3}}

% Definitions des parametres de la page titre
\TitreProjet{Rapport de laboratoire 0}                         % Titre du projet
\TitreRapport{Systèmes de communications}       % Titre du rapport
\Destinataire{M. Jean-Yves Chouinard}         % Nom(s) du destinataire
\TableauMembres{%                                     % Tableau des membres de l'equipe
   910\,055\,897  & Daniel Thibodeau \\\hline
   910\,097\,879  & Francis Valois \\\hline        % matricule & nom & \\\hline
           % matricule & nom & \\\hline     % matricule & nom & \\\hline
}
\DateRemise{21 septembre 2012}                           % Date de remis


% Corps du document

\begin{document}

%   Chapitres
%!TEX root = ../rapport.tex
%!TEX encoding = UTF-8 Unicode

% Chapitres "Introduction"

% modifié par Francis Valois, Université Laval
% 31/01/2011 - version 1.0 - Création du document

\chapter{Introduction}
\label{s:introduction}

%%!TEX root = ../rapport.tex
%!TEX encoding = UTF-8 Unicode

% Chapitres "Introduction"

% modifié par Francis Valois, Université Laval
% 31/01/2011 - version 1.0 - Création du document

\chapter{Préparation}
\label{s:experimentation}
\section{Question 1}


%!TEX root = ../rapport.tex
%!TEX encoding = UTF-8 Unicode
\chapter{Expérimentation, Analyse, Résultat}


%!TEX encoding = UTF-8 Unicode
%!TEX root = ../rapport.tex
% Chapitres "Conclusion"

% modifié par Francis Valois, Université Laval
% 31/01/2011 - version 1.0 - Création du document

\chapter{Conclusion}
\label{s:conclusion}

Au cours de ce septième laboratoire, nous nous sommes familiarisé avec l'implantation pratique d'un oscillateur triangulaire pouvant être employé afin de moduler un signal quelconque. Par ailleurs, nous nous sommes aussi familiarisé avec l'implantation d'un comparateur à hystérésis capable d'effectuer ladite modulation en procurant une protection additionnelle contre le buit. Aussi, nous avons su implanté de manière pratique un filtre Butterworth d'ordre 2 ayant pour fonction d'effectuer la démodulation du signal. De par la précision la similitude entre l'onde de sortie et l'onde d'entrée, nous avons pu constater l'efficacité de notre montage et les qualités de la réponse en fréquence du filtre Butterworth. Il est intéressant de noter le déphasage des signaux de sortie, déphasage tout de même significatif, qui provient de la fonction de filtrage qui induit un déphasage non nul dans sa fonction de transfert.

\appendix
%!TEX root = ../rapport.tex
%!TEX encoding = UTF-8 Unicode
% Chapitres "Annexes"

% modifié par Francis Valois, Université Laval
% 31/01/2011 - version 1.0 - Création du document
\chapter{Annexes}
\label{s:annexes}




\end{document}
% Fin du document

