%!TEX root = ../rapport.tex
%!TEX encoding = UTF-8 Unicode

\chapter{Préparation}
\section*{Numéro 1}
\subsection*{a)}
Sachant que la fréquence d'échantillonnage minimale est donnée par :
\begin{equation}
	f_s \geq 2f_{max}
\end{equation}
il est possible de trouver que la fréquence de Nyquist est de 2000 Hz lorsque la fréquence du signal est de 1000 Hz.
\subsection*{b)}
Sachant que le signal est une sinusoïde de 20 V d'amplitude crête à crête, alors $V_{CAN}$ doit avoir une plage minimale de 20 V.
\subsection*{c)}
Pour trouver le rapport signal à bruit, il suffit de trouver dans l'ordre, la résolution, le bruit de quantification normalisé et la puissance du signal normalisée pour obtenir les valeurs nécessaires.

La résolution s'obtient comme suit:
\begin{equation}
	q = \frac{V_{CAN}}{2^n} = \frac{20}{2^8} = 0.078125 V
\end{equation}

Le bruit de quantification comme suit :
\begin{equation}
	N_Q = \frac{q^2}{12} = \frac{0.078125^2}{12} = 5.0862 \cdot 10^{-4} W
\end{equation}

La puissance normalisée au travers d'une résistance de 1 $\ohm$ est donnée par l'équation suivante :
\begin{equation}
	P_{norm} = \frac{V_{in}^2}{R} = \frac{10^2}{1} = 100 W
\end{equation}

Finalement, le ratio du signal à bruit est donnée par l'équation suivante :
\begin{equation}
	SNR_Q = \frac{P_{norm}}{N_Q} = \frac{400}{5.0862 \cdot 10^{-4}} = 1.966 \cdot 10^5
\end{equation}

Pour obtenir la valeur en dB, il suffit d'utiliser l'équation suivante :
\begin{equation}
	SNR_{Q_{dB}} = 10 \log(SNR_Q) = 10 \log(1.966 \cdot 10^5) = 52.94 dB
\end{equation}

\subsection*{d)}
Le débit peut-être obtenu à l'aide de l'équation suivante:
\begin{equation}
	D = F_s \cdot n = 2000 \cdot 8 = 16 kBits
\end{equation}

\section*{Numéro 2}
\section*{Numéro 3}
Nous savons que :
\begin{equation}
	V_{signal} = \frac{2^nq}{2}
\end{equation}

\begin{equation}
	N_q = \frac{q^2}{12}
\end{equation}

\begin{equation}
	P_{signal} = V_{signal}^2 = \frac{4^nq^2}{4}
\end{equation}

Ainsi,

\begin{equation}
	SNR_{Q_{dB}} = 10 \log\left(\frac{P_{signal}}{N_q}\right) = 10 \log\left(\frac{\frac{4^nq^2}{4}}{\frac{q^2}{12}}\right)
\end{equation}

\begin{equation}
	SNR_{Q_{dB}}= 10 \log(3\cdot4^n) = 10\log(3) + 10\log(4^n) = 10\log(3) + 10n\log(4) \approx 4.77 + 6.02n
\end{equation}

Nous remarquons ici que l'approximation donnée dans l'énoncé du laboratoire est erronée et que notre preuve donne le même résultat que dans le livre.


